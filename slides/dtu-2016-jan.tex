\documentclass[lualatex]{beamer}

\usepackage{fontspec}
\usepackage{fancyvrb}
\usefonttheme{professionalfonts}%  don't change fonts inside beamer
\setmainfont{Lucida Grande}
\setsansfont{Lucida Grande}
\usepackage{unicode-math}
\setmathfont{Asana Math}

\author{Jesper Louis Andersen\\jesper.louis.andersen@gmail.com\\Shopgun Aps}
\date{\today{}}
\title{Real World QuickCheck\\DTU}

\begin{document}

\maketitle

\begin{frame}
\frametitle{Who is Jesper?}
\begin{itemize}
\item Functional programming, type theory, semantics
\item 10 years of Erlang experience
\item 5+ years with QuickCheck
\item Works for a small startup, ShopGun Aps.,
\item BA in CS from Copenhagen University
\end{itemize}
\end{frame}

\begin{frame}
\frametitle{Erlang — in one slide}
Erlang is an (untyped) functional language, with three main focal points:
\begin{itemize}
\item Communication—Concurrency, Protocol handling
\item Robustness—Self-healing, Hardware failure tolerance, soft-realtime
\item Continuous operation—Hot code loading, late binding, dynamic reconfiguration
\end{itemize}

Built originally for telecommunications, in 1986.

Most of the Erlang in this talk is by osmosis.
\end{frame}

\begin{frame}[fragile]
\frametitle{QuickCheck — summary}
We would like to prove a property, such as
\begin{equation}
\begin{split}
	\forall arr \in \text{array}(byte) :\\
	\quad \quad unzip(zip(arr)) = arr
\end{split}
\end{equation}
\begin{itemize}
\item Proof—requires lemmas and theorems about \emph{zip/unzip}, exploiting their structure
\item Check all arrays up to size $n$ is extremely time-consuming, even on large clusters
\item Quickcheck—Generate samples of $arr$ via highly skewed distributions
\end{itemize}
\end{frame}

\begin{frame}[fragile]
\frametitle{QuickCheck - summary}
Erlang QuickCheck notation for the example would be:
\begin{verbatim}
prop_zip_inverse() ->
    ?FORALL(Arr, list(choose(0, 255)),
        begin
            Bin = list_to_binary(Arr),
            Z = zlib:uncompress(zlib:compress(Bin)),
            measure(array_length, length(Arr),
                equals(Bin, Z))
        end).
\end{verbatim}
Example output:
\begin{verbatim}
prop_zip_inverse: … OK, passed 100 tests
array_length:
    Count: 100   Min: 0   Max: 10   Avg: 2.850
    StdDev: 2.350   Total: 285
\end{verbatim}
\end{frame}

\begin{frame}[fragile]
\frametitle{QuickCheck - summary}
``Program testing can be used to show the presence of bugs, but never to show their absence!'' — Dikjstra EWD249 1970

\begin{itemize}
\item Do \emph{not} generate a uniform distribution.
\item Use heuristics and target where the needle/bug is most likely to be in the haystack
\item Generate the nastiest examples which are unlikely to occur in normal operation
\item Once a counterexample is found, employ a \emph{shrinking} strategy to search for a smaller one
\item The power stems from statistical amplification and turning clock cycles into error-finding
\end{itemize}

Doing this well is (part of) the secret, which is the intellectual property of e.g., Quviq's commercial ``Erlang QuickCheck'' implementation.
\end{frame}

\begin{frame}[fragile]
\frametitle{Real world QuickCheck — Cost/Benefit analysis}
Major software constraints:
\begin{itemize}
\item Productivity—Adding features and improving the software
\item Correctness—The software informally does the ``right thing™''
\end{itemize}
Being incorrect to ``too correct'' hampers productivity, so often an equilibrium is used. QuickCheck shines for:
\begin{itemize}
\item Core components—where failure is less tolerated
\item Error kernels—the parts of a system which \emph{absolutely has to be correct}
\end{itemize}
Erlang robustness secret: isolate the error kernel in the program from unstable parts, much like in an OS.
\end{frame}

\begin{frame}[fragile]
\frametitle{Real world QuickCheck — Typical projects}
Erlang core components:
\begin{itemize}
\item Circuit breakers—protecting systems against cascading dependency failure
\item Load regulation—protecting systems from overload
\item Standard libraries—Erlang maps (this talk!)
\item Compilers (Quviq)—by generating random programs
\end{itemize}
Distributed systems:
\begin{itemize}
\item Distributed Hash Tables—millions of potential nodes
\item Database kernels (Basho)—Safe storage is paramount
\end{itemize}
Embedded hardware:
\begin{itemize}
\item AUTOSAR components (Quviq)—Simulate a car bus
\item Walkie-Talkie base stations (Motorola DK)
\end{itemize}
\end{frame}

\begin{frame}
\frametitle{Erlang Maps}
\begin{itemize}
\item New feature in release 17
\item Heterogenous, in contrast to a statically typed OCaml
\item Persistent (functional) data structure
\item Supports the usual \texttt{put, get} operations
\item Syntax support in the compiler, opcode support in the emulator
\item Sugar for an underlying \texttt{maps}-module, with a C implementation for speed
\end{itemize}
Note: OCaml has no built-in support in the language, but has them in modules
\end{frame}

\begin{frame}[fragile]
\frametitle{Erlang Maps—Syntax example}
\begin{verbatim}
2> M = #{ 42 => "foo" }.
#{42 => "foo"}
3> M2 = M#{ 'id' => fun(X) -> X + 1 end }.
#{42 => "foo",id => #Fun<erl_eval.6.54118792>}
5> M.
#{42 => "foo"}
11> case M2 of #{ 42 := V } -> V end.
"foo"
\end{verbatim}
\end{frame}

\begin{frame}
\frametitle{R17 Representation}
\begin{itemize}
\item 2 arrays of erlang terms (at the C layer)
\item Keys and Values, index ordered by key
\item Keys are shared over maps if possible
\item Operations are generally $O(n)$ with binary searching $O(\lg n)$
\item \emph{Extremely} efficient on small maps
\item \emph{Miserable} for large maps
\end{itemize}
\end{frame}

\begin{frame}
\frametitle{Evolution}
\begin{itemize}
\item Release 18 switches the underlying representation as the map grows
\item Based on Phil Bagwell's ``Ideal Hash Trees'' paper from 2000
\item Also known as HAMT (Hash Array Mapped Trie)
\item Key characteristic: \emph{hybrid} between a hash table and a tree
\item Persistent, not ephemeral
\item Memory efficient on modern architectures
\item uses POPCNT CPU instruction (Hamming Weight)
\end{itemize}
\end{frame}

\begin{frame}
\frametitle{Why}
\begin{itemize}
\item Big-Oh, only tells you so much
\item Modern efficiency is memory bound (in practice)
\item Typical DRAM hit: 60–120ns
\item Binary tree: $lg_2(N)$ memory reads
\item Hash table: 1-2 memory reads (hopscotch hashing)
\item HAMT: $lg_{32}(N)$ memory reads
\end{itemize}
Less internal waste than a hash-table or tree.
\end{frame}

\begin{frame}[fragile]
\frametitle{Price}
\begin{itemize}
\item Far more complex code
\end{itemize}

The file \texttt{erl\_maps.c}:\\
\begin{center}
\begin{tabular}[c]{r|l}
Release&Code Size\\
\hline
17.5.6.3 & $552$ \\
18.0.3 & $2233$ \\
\end{tabular}
\end{center}
Not the full story: maps code in other C files as well
\end{frame}

\begin{frame}
\frametitle{But…, is it correct? Cost/Benefit}
Justification for a non-Academic:
\begin{itemize}
\item A faster data structure has \emph{no} value if it is wrong.
\item Has to survive the real world, not just a benchmark in a paper.
\item Erlang/OTP is a language of robustness and stability, not execution efficiency
\item Maps is a central component, harmful faults in maps will hit almost everyone.
\end{itemize}

In other words, this is a prime candidate for formal methods.
\end{frame}

\begin{frame}
\frametitle{Strategy}
\begin{itemize}
\item Not testing the compiler, target the \texttt{maps} module
\item Start with a stateless model
\item Tighten the specification by moving to a stateful model, explore
\item We have no specification, so go in reverse, deriving one with QuickCheck!
\item Above all, have fun doing it!
\end{itemize}
\end{frame}

\begin{frame}[fragile]
\frametitle{Generating terms}
\begin{itemize}
\item Maps keys/values can be arbitrary Erlang terms
\item Build a generator which covers broadly (funs, …)
\end{itemize}

\begin{verbatim}
evil_real() ->
   frequency([
     {20, real()},
     {1, return(0.0)},
     {1, return(0.0/-1)}]).
\end{verbatim}

IEEE 754 Real's have two bit-representations of zero where
$$
	0.0 = -0.0
$$

Bug in internal map compare and \texttt{phash2/1} as well, fixed in R18.
\end{frame}

\begin{frame}[fragile]
\frametitle{Generating terms}
\begin{Verbatim}[fontsize=\small]
map_term() ->
    ?SIZED(Sz, map_term(Sz)).

map_term(0) ->
    frequency([
       {100, oneof([int(), largeint(), evil_real(), …])},
       {10, ?SHRINK(
           oneof([function0(int()), function2(int())]),
           [foo])}
       {10, eqc_gen:largebinary()}
    ]);
map_term(K) ->
    frequency([
        {40, map_term(0)},
        {1, ?LAZY(list(map_term(K div 8)))},
        {1, ?LAZY(?LET(L, list(map_term(K div 8)),
            list_to_tuple(L)))},
        {1, ?LAZY(eqc_gen:map(map_term(K div 8),
            map_term(K div 8)))}
    ]).
\end{Verbatim}
\end{frame}

\begin{frame}[fragile]
\frametitle{Stateless model}

Totality(1):
\begin{equation}
	\forall x \in X : \texttt{doesnt\_crash}[f(x)]
\end{equation}

Inverses:
\begin{equation}
	\forall x \in X : f^{-1}(f(x)) = x
\end{equation}

Idempotence:
\begin{equation}
	\forall x \in X : f(f(x)) = f(x)
\end{equation}

The QuickCheck bag-of-tricks requires you to be able to identify these in real programs
\end{frame}

\begin{frame}[fragile]
\frametitle{Stateless model}
\begin{itemize}
\item Use \texttt{to\_list/1} and \texttt{from\_list/1} as inverses
\item Use \texttt{term\_to\_binary/?} and \texttt{binary\_to\_term/?} (serialization)
\end{itemize}

\begin{verbatim}
opts() ->
  ?LET({Compressed, MinorVersion},
    {oneof([
        [], [compressed], [{compressed, choose(0,9)}]]),
     oneof([
        [], [{minor_version, choose(0,1)}]])},
       Compressed ++ MinorVersion).
\end{verbatim}

\end{frame}

\begin{frame}[fragile]
\frametitle{Stateless model}

\begin{verbatim}
prop_binary_iso_fg() ->
    ?FORALL([Opts, M],
      [opts(),
       oneof([
           eqc_gen:map(
               maps_eqc:map_key(),
               maps_eqc:map_value()),
           maps_eqc:map_map()]) ],
         begin
           Binary = term_to_binary(M, Opts),
           M2 = binary_to_term(Binary),
           equals(M, M2)
         end).
\end{verbatim}

\end{frame}

\begin{frame}[fragile]
\frametitle{Stateless model}

\begin{verbatim}
13> eqc:quickcheck(maps_iso_eqc:prop_binary_iso_fg()).
....Failed! After 5 tests.
#{-3878269413 => hill,
  -1 => stone,
  #Fun<eqc_gen.133.121384563> => sand,
  <<3:2>> => 0.0} /= 
#{-1 => stone,
  -3878269413 => hill,
  #Fun<eqc_gen.133.121384563> => sand,
  <<3:2>> => 0.0}
Shrinking xxxx.x.xxxxxxxx…(35 times)
\end{verbatim}
\end{frame}

\begin{frame}[fragile]
\frametitle{Stateless model}

\begin{verbatim}
#{-1 => flower,2147483647 => flower} /=
#{2147483647 => flower,-1 => flower}
false
\end{verbatim}

\begin{itemize}
\item Wrong order, $2147483647 = 2^{31}-1$
\item Compare function in C layer was wrong
\item Affects R17, have not even started on R18 yet!
\item Fix: OTP-12623
\end{itemize}

\end{frame}

\begin{frame}[fragile]
\frametitle{Stateful models—Why}

\begin{itemize}
\item The \texttt{maps} module has 24 functions, which interact
\item We could generate a list of commands as a trace
\item Then we could both \emph{interpret} the trace and \emph{replay} it against the system-under-test
\item This would give us a random map, generated by the 24 commands, which we could observe for correctness.
\end{itemize}

\begin{itemize}
\item Key idea of stateful models: base the current command and input to that command on the commands that came before it in the trace.
\item I.e., let commands depend on each other.
\end{itemize}
\end{frame}

\begin{frame}[fragile]
\frametitle{Stateful models—Why}

Suppose we have
\begin{Verbatim}
	maps:remove(K, Map)
\end{Verbatim}

\begin{itemize}
\item How often will \texttt{K} be in the map? Rarely.
\item A stateful model tracks what is currently in the map
\item So we can explicitly pick a \texttt{K} which is present
\end{itemize}

\begin{itemize}
\item \texttt{maps:put(K, V, Map)} should try to overwrite existing keys
\item If we know we are overwriting a key, we get better code coverage
\item Can also handle side-effects
\end{itemize}

Stateful models allows for far tighter specifications, and better coverage, which improves their ability to find counterexamples quickly.
\end{frame}

\begin{frame}[fragile]
\frametitle{Stateful model}

Stateful models are the norm in large developments, often composing several such models together

\begin{itemize}
\item Basic strategy: model the map as $(K * V) \; list$
\item Use a simple data structure in the model with worse asymptotics (standard trick)
\end{itemize}

Alternative model: simplify and only \emph{count} the elements in the map.

\end{frame}

\begin{frame}[fragile]
\frametitle{Primary property (slightly simplified)}
\begin{Verbatim}[fontsize=\small]
map_property() ->
      ?FORALL(Cmds, commands(?MODULE),
        begin
          {H,S,R} = run_commands(?MODULE, Cmds),
          
          collect(eqc_lib:summary('Final map size'),
          	model_size(S),
          collect(eqc_lib:summary('Command Length'),
          	length(Cmds),
          aggregate(with_title('Commands'),
          	command_names(Cmds),
          aggregate(with_title('Features'),
          	call_features(H),
            pretty_commands(?MODULE, Cmds, {H,S,R}, R == ok)))))
        end).
\end{Verbatim}
\end{frame}

\begin{frame}
What is going on
\begin{itemize}
\item Generate trace symbolically $Cmds := S_1 \to S_2 \to S_3 \to …$
\item Run the trace against the system
\item Note: command generation \emph{never} invokes the system-under-test
\end{itemize}
Each command is specified by its:
\begin{itemize}
\item Precondition—in what state and with what arguments can the command fire?
\item Argument-generator—what arguments should be generated for the command?
\item Next State—if the command is executed, how is the model state altered?
\item Postcondition/Return—If replayed against the system-under-test, what is the expected return value?
\end{itemize}
\end{frame}

\begin{frame}[fragile]
\frametitle{Stateful model}

\begin{verbatim}
put(Key, Value) ->
    maps_runner:put(Key, Value).
    
put_args(State) ->
    [map_key(State), map_value()].
\end{verbatim}
\end{frame}

\begin{frame}[fragile]
\frametitle{Stateful model}
\begin{Verbatim}[fontsize=\small]
put_next(#state { contents = C } = State, _, [K, V]) ->
    State#state { contents = add_contents(K, V, C) }.

put_return(#state { contents = C}, [K, V]) ->
    maps:from_list(add_contents(K, V, C)).

put_features(S, [K, _Value], _Res) ->
    case member(K, S) of
        true -> ["R003: put/3 on existing key"];
        false -> ["R004: put on a new key"]
    end.
\end{Verbatim}
\end{frame}

\begin{frame}[fragile]
\frametitle{Stateful model}
\begin{Verbatim}[fontsize=\small]
is_key_args(#state { contents = C } = State) ->
    frequency(
        [{10, ?LET(Pair, elements(C),
        		[element(1, Pair)])} || C /= [] ] ++
        [{1, ?SUCHTHAT([K], [map_key(State)],
        		find(K, 1, C) == false)}]).

is_key_features(_S, [_K], true) ->
	["R001: is_key/2 on a present key"];
is_key_features(_S, [_K], false) ->
	["R002: is_key/2 on a non-existing key"].
\end{Verbatim}
\end{frame}

\begin{frame}[fragile]
\frametitle{Stateful model}
\begin{Verbatim}[fontsize=\small]
m_get_args(#state { contents = C } = S) ->
    frequency(
      [{5, ?LET(Pair, elements(C),
      		[element(1, Pair)])} || C /= []] ++
      [{1, ?SUCHTHAT([K], [map_key(S)],
      		find(K,1,C) == false)}]).

m_get_return(#state { contents = C } = S, [K]) ->
    case find(K,1,C) of
       false -> badkey(S, K);
       {K, V} -> V
    end.
\end{Verbatim}
\end{frame}

\begin{frame}[fragile]
\frametitle{Stateful model}
{\small
\begin{verbatim}
Features
9.9% {{maps_eqc,merge,1},"R019: Merging two maps"}
9.0% {{maps_eqc,put,2},"R004: put on a new key"}
7.7% {{maps_eqc,update,2},"R015: update/3 on an existing key"}
7.7% {{maps_eqc,remove,1},"R011: remove/2 of present key"}
5.4% {{maps_eqc,to_list,0},"R013: to_list/1 called on map"}
5.0% {{maps_eqc,values,0},"R014: values/1 called on map"}
4.5% {{maps_eqc,keys,0},"R010: Calling keys/1 on the map"}
4.1% {{maps_eqc,m_get,1},"R023: get on a successful key"}
…
\end{verbatim}
}

There are 55 features in total.
\end{frame}

\begin{frame}
\frametitle{Large maps}

\begin{itemize}
\item Problem: We need maps of 25000 random elements for the keys to collide
\item Observation: This is not feasible
\item Solution: Pre-generate a set of siblings!
\end{itemize}

Prefer siblings when inserting new nodes. Forces collisions on small maps!

\end{frame}

\begin{frame}
\frametitle{When Erlang crashes}

\begin{itemize}
\item Problem: Running test cases crashes Erlang
\item Observation: Maps runs in a separate process
\item Solution: Distribute!
\end{itemize}

Run the map in one Erlang node, under \texttt{heart(1)}, the EQC model in another node.

\begin{itemize}
\item Crash! Wait for heart restart
\item Continue shrinking
\item Model is safe in its own node
\end{itemize}

\end{frame}

\begin{frame}[fragile]
\frametitle{Problems found}

\begin{verbatim}
Thread 1 (Thread 0x7fd638fbf700 (LWP 7208)):
#0  0x00007fd67aae54b7 in raise () from libc.so.6
#1  0x00007fd67aae688a in abort () from libc.so.6
#2  0x0000000000628232 in erl_assert_error
    (expr=0x73171b "n > 1", 
     func=0x731fb0 <…> "hashmap_from_chunked_array", 
     file=0x731600 "beam/erl_map.c", line=682)
       at sys/unix/sys.c:2908
\end{verbatim}
\end{frame}

\begin{frame}[fragile]
\frametitle{Problems found}

\begin{verbatim}
'eqc@127.0.0.1' got a corrupted external term from
  'runner@127.0.0.1' on distribution channel 15879
\end{verbatim}

\end{frame}

\begin{frame}[fragile]
\frametitle{Problems Solved}

\begin{itemize}
\item Common bug: wrong sign extension in callback from \texttt{qsort(3)} in C
\item Many bugs were fixed inside Ericssons OTP team using the model
\item Rough bug count: 10-15
\item Typical fixes: 6492bf359, e3f21d4911
\end{itemize}

Twitter: ``The QuickCheck model by @jlouis666 has been very helpful finding subtle bugs in our Maps code. You have my deepest thanks for your work!''

\end{frame}


\begin{frame}
\frametitle{Insights}

\begin{itemize}
\item Implementation is mostly mechanic
\item Sequential \texttt{eqc\_statem} is enough!
\item Exploratory process was a success
\item Covers 85\% of the C layer
\end{itemize}

\end{frame}

\begin{frame}
\frametitle{Open Source}

See \url{https://github.com/jlouis/maps_eqc}

\end{frame}

\end{document}
