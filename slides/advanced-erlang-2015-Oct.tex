\documentclass[lualatex]{beamer}

\usepackage{fontspec}
\usefonttheme{professionalfonts}%  don't change fonts inside beamer
\setmainfont{Lucida Grande}
\setsansfont{Lucida Grande}
\usepackage{unicode-math}
\setmathfont{Asana Math}

\author{Jesper Louis Andersen\\jesper.louis.andersen@gmail.com\\Shopgun Aps}
\date{\today{}}
\title{Testing Erlang/OTP Release 18 maps\\Experience report}

\begin{document}

\maketitle

\begin{frame}
\frametitle{Erlang Maps}
\begin{itemize}
\item New feature in release 17
\item Supports the usual \texttt{put, get} operations
\item Syntax support in the compiler, opcode support in the emulator
\item Driven by a new set of BIFs, \texttt{maps}-module
\end{itemize}
\end{frame}

\begin{frame}
\frametitle{R17 Representation}
\begin{itemize}
\item 2 arrays of erlang terms (at the C layer)
\item Keys and Values, index ordered by key
\item Keys are shared over all maps if using the $:=$ binding operator
\item Operations are generally $O(n)$ with binary searching $O(\lg n)$
\item \emph{Extremely} efficient on small maps
\item \emph{Miserable} for large maps
\end{itemize}
\end{frame}

\begin{frame}
\frametitle{Evolution}
\begin{itemize}
\item Release 18 switches the underlying representation as the map grows
\item Based on Phil Bagwell's ``Ideal Hash Trees'' paper from 2000
\item Also known as HAMT (Hash Array Mapped Trie)
\item Key characteristic: hybrid between a hash table and a tree
\item Persistent, not ephemeral
\item Memory efficient on modern architectures
\item uses POPCNT CPU instruction (Hamming Weight)
\end{itemize}
\end{frame}

\begin{frame}
\frametitle{Why}
\begin{itemize}
\item Big-Oh, only tells you so much
\item Modern algorithms efficiency are memory bound (in practice)
\item Typical DRAM hit: 120ns
\item Binary tree: $lg_2(N)$ memory reads
\item Hash table: 1-2 memory reads (hopscotch hashing)
\item HAMT: $lg_{32}(N)$ memory reads
\end{itemize}
Less internal waste than a hash-table or tree.
\end{frame}

\begin{frame}[fragile]
\frametitle{Price}
\begin{itemize}
\item Far more complex code
\end{itemize}

The file \texttt{erl\_maps.c}:\\
\begin{center}
\begin{tabular}[c]{r|l}
Release&Code Size\\
\hline
17.5.6.3 & $552$ \\
18.0.3 & $2233$ \\
\end{tabular}
\end{center}
Not the full story: maps code in other C files as well
\end{frame}

\begin{frame}
\frametitle{But…, is it correct? Cost/Benefit}
Justification for a non-Academic:
\begin{itemize}
\item A faster data structure has \emph{no} value if it is wrong.
\item Has to survive the real world, not just a benchmark in a paper.
\item Erlang/OTP is a language of robustness and stability, not execution efficiency
\item Maps is a central component, harmful faults in maps will hit almost everyone.
\end{itemize}

In other words, this is a prime candidate for formal methods.
\end{frame}

\begin{frame}
\frametitle{Strategy}
\begin{itemize}
\item Not testing the compiler, target the \texttt{maps} module
\item Start with a stateless model
\item Tighten the specification by moving to a stateful model
\item See where this takes us
\item Above all, have fun doing it!
\end{itemize}

Modeling is \emph{exploratory} in nature
\end{frame}

\begin{frame}[fragile]
\frametitle{Generating terms}
\begin{itemize}
\item Maps keys/values can be arbitrary Erlang terms
\item Build a generator which covers broadly (funs, …)
\end{itemize}

\begin{verbatim}
evil_real() ->
   frequency([
     {20, real()},
     {1, return(0.0)},
     {1, return(0.0/-1)}]).
\end{verbatim}

IEEE 754 Real's have two bit-representations of zero where
$$
	0.0 = -0.0
$$

Bug in internal map compare and \texttt{phash2/1} as well, fixed in R18.
\end{frame}

\begin{frame}[fragile]
\frametitle{Stateless model}
\begin{itemize}
\item Use \texttt{to\_list/1} and \texttt{from\_list/1} as inverses
\item Use \texttt{term\_to\_binary/?} and \texttt{binary\_to\_term/?}
\end{itemize}

\begin{verbatim}
opts() ->
  ?LET({Compressed, MinorVersion},
    {oneof([
        [], [compressed], [{compressed, choose(0,9)}]]),
     oneof([
        [], [{minor_version, choose(0,1)}]])},
       Compressed ++ MinorVersion).
\end{verbatim}

\end{frame}

\begin{frame}[fragile]
\frametitle{Stateless model}

\begin{verbatim}
prop_binary_iso_fg() ->
    ?FORALL([Opts, M],
      [opts(),
       oneof([
           eqc_gen:map(
               maps_eqc:map_key(),
               maps_eqc:map_value()),
           maps_eqc:map_map()]) ],
         begin
           Binary = term_to_binary(M, Opts),
           M2 = binary_to_term(Binary),
           equals(M, M2)
         end).
\end{verbatim}

\end{frame}

\begin{frame}[fragile]
\frametitle{Stateless model}

\begin{verbatim}
13> eqc:quickcheck(maps_iso_eqc:prop_binary_iso_fg()).
....Failed! After 5 tests.
#{-3878269413 => hill,
  -1 => stone,
  #Fun<eqc_gen.133.121384563> => sand,
  <<3:2>> => 0.0} /= 
#{-1 => stone,
  -3878269413 => hill,
  #Fun<eqc_gen.133.121384563> => sand,
  <<3:2>> => 0.0}
Shrinking xxxx.x.xxxxxxxx…(35 times)
\end{verbatim}
\end{frame}

\begin{frame}[fragile]
\frametitle{Stateless model}

\begin{verbatim}
#{-1 => flower,2147483647 => flower} /=
#{2147483647 => flower,-1 => flower}
false
\end{verbatim}

\begin{itemize}
\item Wrong order, $2147483647 = 2^{31}-1$
\item Compare function in C layer was wrong
\item Affects R17, have not even started on R18 yet!
\item Fix: OTP-12623
\item This is Textbook QuickCheck Chapter 1!
\end{itemize}

\end{frame}

\begin{frame}[fragile]
\frametitle{Stateful model}

\begin{itemize}
\item Basic strategy: represent the map as $(K * V) \; list$
\item Use a simple data structure in the model with worse asymptotics
\item 24 Commands, each command updates the list accordingly
\item Test persistence by keeping snapshots of the map and list
\item Recall old snapshots
\item Run the map in a separate process
\end{itemize}

The choice of having a process run the map was completely arbitrary, but proved valuable (see upcoming slides).

\end{frame}

\begin{frame}[fragile]
\frametitle{Stateful model}

\begin{verbatim}
update(K, V) ->
    case maps_runner:update(K, V) of
        {error, Reason} -> {error, Reason};
        M -> M
    end.
\end{verbatim}
\end{frame}

\begin{frame}[fragile]
\frametitle{Stateful model}
\begin{verbatim}
update_args(#state { contents = C } = State) ->
    frequency(
      [{5,
        ?LET(Pair, elements(C),
          [element(1, Pair), map_value()])} || C /= [] ]
    ++
      [{1,
        ?SUCHTHAT([K, _V], [map_key(State), map_value()],
          find(K, 1, C) == false)}]).
\end{verbatim}
\end{frame}

\begin{frame}[fragile]
\frametitle{Stateful model}
\begin{verbatim}
update_next(#state { contents = C } = S, _, [K, V]) ->
    S#state {
      contents = replace_contents(K, V, C)
    }.

update_return(#state { contents = C} = S, [K, V]) ->
    case member(K, S) of
        true ->
          maps:from_list(replace_contents(K, V, C));
        false ->
          badarg(S, K)
    end.
\end{verbatim}
\end{frame}

\begin{frame}[fragile]
\frametitle{Stateful model}
\begin{verbatim}
update_features(S, [K, _], _) ->
   case member(K, S) of
        true -> {'r015', update, existing_key};
        false -> {'r016', update, non_existing_key}
   end.
\end{verbatim}
\end{frame}

\begin{frame}[fragile]
\frametitle{Stateful model}
{\small
\begin{verbatim}
Features
9.9% {{maps_eqc,merge,1},"R019: Merging two maps"}
9.0% {{maps_eqc,put,2},"R004: put on a new key"}
7.7% {{maps_eqc,update,2},"R015: update/3 on an existing key"}
7.7% {{maps_eqc,remove,1},"R011: remove/2 of present key"}
5.4% {{maps_eqc,to_list,0},"R013: to_list/1 called on map"}
5.0% {{maps_eqc,values,0},"R014: values/1 called on map"}
4.5% {{maps_eqc,keys,0},"R010: Calling keys/1 on the map"}
4.1% {{maps_eqc,m_get,1},"R023: get on a successful key"}
…
\end{verbatim}
}

There are 55 features in total.
\end{frame}

\begin{frame}
\frametitle{Large maps}

\begin{itemize}
\item Problem: We need maps of 25000 random elements for the keys to collide
\item Observation: This is not feasible
\item Solution: Pre-generate a set of siblings!
\end{itemize}

Prefer siblings when inserting new nodes. Forces collisions on small maps!

\end{frame}

\begin{frame}
\frametitle{When Erlang crashes}

\begin{itemize}
\item Problem: Running test cases crashes Erlang
\item Observation: Maps runs in a separate process
\item Solution: Distribute!
\end{itemize}

Run the map in one Erlang node, under \texttt{heart(1)}, the EQC model in another node.

\begin{itemize}
\item Crash! Wait for heart restart
\item Continue shrinking
\item Model is safe in its own node
\end{itemize}

\end{frame}

\begin{frame}[fragile]
\frametitle{Problems found}

\begin{verbatim}
Thread 1 (Thread 0x7fd638fbf700 (LWP 7208)):
#0  0x00007fd67aae54b7 in raise () from libc.so.6
#1  0x00007fd67aae688a in abort () from libc.so.6
#2  0x0000000000628232 in erl_assert_error
    (expr=0x73171b "n > 1", 
     func=0x731fb0 <…> "hashmap_from_chunked_array", 
     file=0x731600 "beam/erl_map.c", line=682)
       at sys/unix/sys.c:2908
\end{verbatim}
\end{frame}

\begin{frame}[fragile]
\frametitle{Problems found}

\begin{verbatim}
'eqc@127.0.0.1' got a corrupted external term from
  'runner@127.0.0.1' on distribution channel 15879
\end{verbatim}

\end{frame}

\begin{frame}[fragile]
\frametitle{Problems Solved}

\begin{itemize}
\item Common bug: wrong sign extension in callback from \texttt{qsort(3)} in C
\item Many bugs were fixed inside Ericssons OTP team using the model
\item Rough bug count: 10-15
\item Typical fixes: 6492bf359, e3f21d4911
\end{itemize}

Twitter: ``The QuickCheck model by @jlouis666 has been very helpful finding subtle bugs in our Maps code. You have my deepest thanks for your work!''

\end{frame}


\begin{frame}
\frametitle{Insights}

\begin{itemize}
\item Implementation is mostly mechanic
\item Sequential \texttt{eqc\_statem} is enough!
\item Exploratory process was a success
\item Covers 85\% of the C layer
\end{itemize}

\end{frame}

\begin{frame}
\frametitle{Open Source}

See \url{https://github.com/jlouis/maps_eqc}

\end{frame}

\end{document}
